\subsubsection{Unix Domain Socket Messaging Format}
\label{ProvProt}
\begin{table}[H]
\begin{tabular}{|l|l|}
\hline
Field Name & Field Type\\
\hline
Message Type & char\\
Message Length & int\\
\hline
\end{tabular}
\caption{Message Header}
\end{table}

Message type may take one of a series of enumerated values. The logical values are SETUP, SYSCALL and AGGREGATE. SETUP and SYSCALL message headers will be followed by a setup or syscall message as detailed below. AGGREGATE message headers are followed by a series of message header, message pairs. The purpose of aggregate messages is to allow both the front and back end to make bulk writes/reads to the socket to gain greater performance.

\begin{table}[H]
\begin{tabular}{|l|l|}
\hline
Field Name & Field Type\\
\hline
EXE Name Length & int\\
CWD Length & int\\
\hline
\multicolumn{2}{|c|}{String Data}\\
\hline
\end{tabular}
\caption{Setup Message}
\end{table}

\begin{table}[H]
\begin{tabular}{|l|l|}
\hline
Field Name & Field Type\\
\hline
syscall Number & int\\
Arguments & int64*6\\
Return & int64\\
Timestamp & int64\\
String Arguments & char\\
\hline
\multicolumn{2}{|c|}{String Data}\\
\hline
\end{tabular}
\caption{Syscall Message}
\end{table}

The string arguments element is a single byte bitfield that indicates which arguments are strings. String arguments will have the character pointer that would be in the argument slot replaced with a length and the string data appended to the end of the message in argument order.

\subsubsection{Provenance Object Format}
\label{POF}
\begin{table}[H]
\begin{tabular}{|l|l|}
\hline
Field Name & Field Type\\
\hline
version & integer\\
user & string\\
type & integer\\
timestamp & datetime\\
eName & string\\
ancestors & list\\
descendants & list \\
\hline
\end{tabular}
\end{table}
